\chapter{Lagrangian Submanifolds}
\section{Tautological Form and Symplectomorphisms}
\begin{exercise}
Let $M$ and $N$ be smooth manifolds, $F : M \to N$ a diffeomorphism and $A \in \Gamma\del[1]{T^{(0,k)}TN}$, $k \in \Zbb$, $k \geq 1$. Then 
\begin{equation}
F^*A(X_1,\dots,X_k) = A(F_*X_1,\dots,F_*X_k) \circ F
\end{equation}
\noindent holds for all $X_1,\dots,X_k \in \Xfrak(M)$.
\label{ex_properties_pullback}
\end{exercise}

\begin{solution}
Let $p \in M$. Then
\begin{align*}
F^*A(X_1,\dots,X_k)(p) &= (F^*A)_p(X_1\vert_p,\dots,X_k\vert_p)\\
&= A_{F(p)}\del[1]{\d F_p(X_1\vert_p),\dots,\d F_p(X_k\vert_p)}\\
&= A_{F(p)}\del[1]{(F_*X_1)_{F(p)},\dots,(F_*X_k)_{F(p)}}\\
&= A\del[1]{F_*X_1,\dots,F_*X_k}\del[1]{F(p)}.
\end{align*}
\end{solution}

\begin{exercise}
~
\begin{enumerate}[label = \textup{(}\alph*\textup{)}]
\item Let $(M,\omega)$ be a symplectic manifold and $\alpha \in \Omega^1(M)$ such that $\omega = -\d \alpha$. Furthermore, let $g : M \to M$ be a diffeomorphism such that $g^*\alpha = \alpha$. Then there exists a unique vector field $X \in \Xfrak(M)$, such that $X \intprod \omega = -\alpha$ and
\begin{equation}
\rho_t \circ g = g \circ \rho_t
\end{equation}
\noindent holds, where $\rho : D \to M$ is the local flow generated by $X$.
\end{enumerate}
\end{exercise}

\begin{solution}
For (a), we observe that $\widehat{\omega} : TM \to T^*M$ is a smooth bundle isomorphism (see \cite[341]{lee:smooth_manifolds:2013}). Thus we define $X : M \to TM$ by 
\begin{equation*}
X := -\widehat{\omega}^{-1}(\alpha).
\end{equation*}
As a composition of smooth maps, $X$ is smooth and clearly, it is a section of the projection $\pi : TM \to M$ by definition. Hence $X \in \Xfrak(M)$. Furthermore $X \intprod \omega = \widehat{\omega}(X) = -\alpha$.\\
Let $\rho$ denote the flow of $X$ and define 
\begin{equation*}
\theta_t := g \circ \rho_t \circ g^{-1}, \qquad t \in \Rbb.
\end{equation*}
Then we have that
\begin{equation*}
\theta_0 = g \circ \rho_0 \circ g^{-1} = g \circ \id_M \circ g^{-1} = \id_M
\end{equation*}
\noindent and for $t \in \Rbb$, $p \in M$
\begin{align*}
\del[1]{\theta^{(p)}}'(t) &= \del[1]{g \circ \rho^{(g^{-1}(p))}}'(t)\\ 
&= \d g_{\rho^{(g^{-1}(p))}(t)}\del[1]{\rho^{(g^{-1}(p))}}'(t)\\
&= \d g_{\rho^{(g^{-1}(p))}(t)} X_{\rho^{(g^{-1}(p))}(t)}\\
&= \del[0]{g_* X}_{g(\rho^{(g^{-1}(p))})(t)}\\
&= \del[0]{g_* X}_{\theta^{(p)}(t)}.
\end{align*}
Since $g* \alpha = \alpha$ and $\omega = -\d \alpha$, we have that $g$ is a symplectomorphism. Indeed, we have that
\begin{equation*}
g^*\omega = g^*(-\d\alpha) = -\d(g^*\alpha) = -\d\alpha = \omega.
\end{equation*}
\noindent by \cite[366]{lee:smooth_manifolds:2013}. Let $Y \in \Xfrak(M)$. Then by exercise \ref{ex_properties_pullback} we have that 
\begin{align*}
\omega(g_*X,Y) \circ g &= (g^*\omega)(X,g_*^{-1}Y)\\
&= \omega\del[1]{X,g_*^{-1}Y}\\
&= (X \intprod \omega)\del[1]{g_*^{-1}Y}\\
&= -\alpha\del[1]{g_*^{-1}Y} \circ g^{-1}\\
&= -(g^*\alpha)\del[1]{g_*^{-1}Y}\\
&= -\alpha(Y) \circ g\\
&= (X \intprod \omega)(Y) \circ g\\
&= \omega(X,Y) \circ g.
\end{align*}
Thus $\omega(g_*X,Y) = \omega(X,Y)$ for all $Y \in \Xfrak(M)$. Since $\widehat{\omega}: \Xfrak(M) \to \Xfrak^*(M)$ is an isomorphism, we get that $g_*X = X$. We deduce that the local flow $\theta$ is also generated by $X$ and thus by uniqueness \cite[212]{lee:smooth_manifolds:2013} we deduce that $\theta = \rho$ which implies $\theta_t = \rho_t$ and thus by definition of $\theta$, $\rho_t \circ g = g \circ \rho_t$.\\
For (b), let $X := X^i \frac{\partial}{\partial x^i} + Y^i \frac{\partial}{\partial \xi^i}$. We calculate
\begin{align*}
X \intprod \omega &= \sum_{i = 1}^n \del[1]{X \intprod (\d x^i \wedge \d \xi^i)}\\
&= \sum_{i = 1}^n \del[1]{(X \intprod \d x^i) \wedge \d y^i) - \d x^i \wedge (X \intprod \d \xi^i)}\\
&= \sum_{i = 1}^n \del[1]{X^i \d \xi^i - Y^i \d x^i}.
\end{align*}
Since $X \intprod \omega = -\alpha$, we get that 
\begin{equation*}
X = \xi^i\frac{\partial}{\partial \xi^i}.
\end{equation*}
Define an isotopy $\rho: \Rbb \times T^*M \to T^*M$ by $\rho(t,p) := (x,e^t\xi)$, where $p = (x,\xi)$. Then we have that $\rho_0 = \id_M$ and 
\end{solution}