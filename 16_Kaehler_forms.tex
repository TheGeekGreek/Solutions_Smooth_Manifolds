\chapter{K\"ahler Forms}
\section{The Fubini-Study Structure}

\begin{lemma}
For every $z \in \Cbb^n$ there exists $A \in \Urm(n)$ such that $Az \in \Cbb \times \cbr[0]{0}^{n-1}$.
\label{lem:rotation}
\end{lemma}

\begin{proof}
If $z = 0$, for example $Iz = 0 \in \Cbb \times \cbr[0]{0}^{n-1}$. So let us assume that $\abs[0]{z} = 1$. Let $(e_\nu)$ denote the standard basis of $\Cbb^n$. Since $z \neq 0$, the set $\cbr[0]{z}$ is linearly independent and thus by exercise B.4 \cite[620]{lee:smooth_manifolds:2013} contained in a basis for $\Cbb^n$, which can be made into an orthonormal basis using the Gram-Schmidt algorithm, say $(\widetilde{e}_\nu)$, where $\widetilde{e}_1 = z$. Define a linear mapping $\widetilde{A} : \Cbb^n \to \Cbb^n$ by matrix multiplication with $\widetilde{A} := (\widetilde{e}_1,\dots,\widetilde{e}_n)$. Clearly $\widetilde{A} e_\nu = \widetilde{e_\nu}$ and $\widetilde{A} \in \Urm(n)$. Thus $\widetilde{A}^{-1} \in \Urm(n)$ and $\widetilde{A}^{-1}z = e_1 \in \Cbb \times \cbr[0]{0}^{n-1}$. Now set $A := \widetilde{A}^{-1}$. If $\abs[0]{z} \neq 1$, we have that 
\begin{equation}
Az = A\del[3]{\abs[0]{z}\frac{z}{\abs[0]{z}}} = \abs[0]{z}A\del[3]{\frac{z}{\abs[0]{z}}} = \abs[0]{z}e_1 \in \Cbb \times \cbr[0]{0}^{n-1}.
\end{equation}
\end{proof}

\begin{exercise}
\begin{enumerate}[label = \textup{(}\alph*\textup{)}]
\item The form
\begin{equation}
\frac{i}{2}\partial \overline{\partial}\log\del[0]{\abs[0]{z}^2 + 1}
\end{equation}
\noindent on $\Cbb^n$ is a K\"ahler form.
\end{enumerate}
\end{exercise}

\begin{solution}
For (a), define a smooth function $\rho: \Cbb^n \to \Rbb$ by 
\begin{equation*}
\rho(z) := \log\del[0]{\abs[0]{z}^2 + 1}.
\end{equation*}
Then we have 
\begin{equation}
\frac{\partial^2 \rho}{\partial z_\mu \partial\overline{z}_\nu} = \frac{\partial}{\partial z_\mu} \frac{z_\nu}{\abs[0]{z}^2 + 1} = \frac{\delta_{\nu\mu}}{\abs[0]{z}^2 + 1} - \frac{z_\nu \overline{z}_\mu}{(\abs[0]{z}^2 + 1)^2}.
\label{eq:derivative}
\end{equation}
Let $A \in \Urm(n)$. Then 
\begin{equation*}
A^*\omega_{\FSrm} = \frac{i}{2}A^*\partial\overline{\partial}\rho = \frac{i}{2}\partial\overline{\partial}A^*\rho = \frac{i}{2}A\partial\overline{\partial}(\rho \circ A) = \frac{i}{2}\partial\overline{\partial}\rho = \omega_{\FSrm}
\end{equation*}
\noindent since
\begin{equation*}
\abs[0]{Az}^2 = \langle Az,Az \rangle = z^tA^t\overline{A}\overline{z} = z^t\overline{z} = \langle z,z \rangle = \abs[0]{z}^2
\end{equation*}
\noindent for any $z \in \Cbb^n$. Let $p := (a,0,\dots,0) \in \Cbb \times \cbr[0]{0}^{n-1}$. From \ref{eq:derivative} we deduce that
\begin{align*}
\frac{\partial^2 \rho}{\partial z_\mu \partial\overline{z}_\nu}(p) = \begin{cases}
\frac{1}{(\abs[0]{a}^2 + 1)^2} & \nu = \mu = 1\\
\frac{1}{\abs[0]{a}^2 + 1} & \nu = \mu > 1\\
0 & \nu \neq \mu
\end{cases}
\end{align*}
Therefore the matrix $\del[1]{\frac{\partial^2 \rho}{\partial z_\mu \partial\overline{z}_\nu}(p)}$ does have the eigenvalues $\frac{1}{(\abs[0]{a}^2 + 1)^2}$ and $\frac{1}{\abs[0]{a}^2 + 1}$ which are both positive. Thus $\del[1]{\frac{\partial^2 \rho}{\partial z_\mu \partial\overline{z}_\nu}(p)}$ is positive definite.\\
For (b), we have that $\varphi \circ \varphi = \id_U$ and $\varphi$ is holomorphic. Furthermore
\begin{align*}
\varphi^*\log\del[0]{\abs[0]{z}^2 + 1} &= \log\del[0]{\abs[0]{\varphi(z)}^2 + 1}\\
&= \log\del[3]{\frac{1}{\abs[0]{z_1}^2}\del[3]{1 + \sum_{\nu = 2}^n\abs[0]{z_\nu}^2} + 1}\\
&= \log\del[3]{\frac{1}{\abs[0]{z_1}^2}\del[0]{\abs[0]{z}^2 + 1}}\\
&= \log\del[0]{\abs[0]{z}^2 + 1} + \log\del[3]{\frac{1}{\abs[0]{z_1}^2}}.
\end{align*}
For (c), we have that 
\begin{align*}
\partial\overline{\partial}\varphi^*\log\del[0]{\abs[0]{z}^2 + 1} &= \partial\overline{\partial}\log\del[0]{\abs[0]{z}^2 + 1} + \partial\overline{\partial}\log\del[3]{\frac{1}{\abs[0]{z_1}^2}}\\
&= \partial\overline{\partial}\log\del[0]{\abs[0]{z}^2 + 1} - \partial\overline{\partial}\log z_1 - \partial\overline{\partial}\log \overline{z}_1\\
&= \partial\overline{\partial}\log\del[0]{\abs[0]{z}^2 + 1}
\end{align*}
\noindent by part (b) and so
\begin{equation*}
\varphi^*\omega_{\FSrm} = \frac{i}{2}\varphi^*\partial\overline{\partial}\rho = \frac{i}{2}\partial\overline{\partial}\varphi^*\rho = \frac{i}{2}\partial\overline{\partial}\rho = \omega_{\FSrm}.
\end{equation*}
For (f), we have that
\begin{align*}
\int_{\CPbb^1} \omega_{\FSrm} &= \int_{\Rbb^2}\frac{\d x \wedge \d y}{(x^2 + y^2 + 1)^2}\\
&= \int_{\Rbb^2}\frac{\d x\d y}{(x^2 + y^2 + 1)^2}\\
&= \int_0^\infty\int_0^{2\pi}\frac{r}{\del[0]{r^2 + 1}^2}\d\theta\d r\\
&= 2\pi \int_0^\infty\frac{r}{\del[0]{r^2 + 1}^2}\d r\\
&= \pi \int_1^\infty \frac{\d s}{s^2}\\
&= \pi.
\end{align*}
\end{solution}