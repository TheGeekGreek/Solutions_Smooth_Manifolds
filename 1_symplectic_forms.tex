\chapter{Symplectic Forms}
\section{Symplectic Linear Algebra}
\begin{exercise}
Let $V$ be a finite dimensional real vector space and $\omega$ be a $2$-covector on $V$. Then $\Omega$ is nondegenerate if and only if for each nonzero $v \in V$ there exists $w \in V$ such that $\omega(v,w) \neq 0$.
\end{exercise}

\begin{solution}
We have that 
\begin{equation*}
\ker \widehat{\omega} = \cbr[0]{v \in V : \forall w \in V\del[0]{\omega(v,w) = 0}}.
\end{equation*}
Hence if $\omega$ is nondegenerate we have that $\widehat{\omega}$ is an isomorphism and thus $\ker\widehat{\omega} = \cbr[0]{0}$. Conversly, we have that $\ker\widehat{\omega} = \cbr[0]{0}$ and since $\dim V = \dim V^*$, we have that $\widehat{\omega}$ is an isomorphism.
\end{solution}

\begin{exercise}
Let $(V,\omega)$ be a symplectic vector space and $S,T \subseteq V$ be linear subspaces.
\begin{enumerate}[label = \textup{(}\alph*\textup{)}]
\item $\dim S + \dim S^\omega = \dim V$.
\item $\del[1]{S^\omega}^\omega = S$.
\item $S \subseteq T \Leftrightarrow T^\omega \subseteq S^\omega$.
\item $\omega\vert_{S}$ nondegenerate $\Leftrightarrow S \cap S^\omega = \cbr[0]{0} \Leftrightarrow V = S \oplus S^\omega$.
\item If $S \subseteq S^\omega$, then $\dim S \leq \frac{1}{2}\dim V$.
\item If $S$ is of codimension $1$, then $S$ is coisotropic. 
\item $S$ lagrangian $\Leftrightarrow$ $S$ isotropic and coisotropic $\Leftrightarrow$ $S = S^\omega$.
\end{enumerate}
\end{exercise}

\begin{solution}
For proving (a), consider the mapping $\Phi : V \to S^*$ defined by $\Phi(v) := \omega(v,\cdot)\vert_S$. Clearly, $\ker \Phi = S^\omega$. Let $\varphi \in S^*$. By exercise B.13 \cite[623]{lee:smooth_manifolds:2013}, there exists an extension $\widehat{\varphi} \in V^*$ of $\varphi$. Since $\widehat{\omega}$ is an isomorphism, there exists $v \in V$ such that $\widehat{\varphi} = \omega(v,\cdot)$. This implies $\widehat{\varphi}\vert_S = \omega(v,\cdot)\vert_S$. Hence we get that $\Phi$ is surjective and thus $\Phi(V) = S^*$. Hence the rank-nullity law \cite[627]{lee:smooth_manifolds:2013} implies that 
\begin{equation*}
\dim V = \dim S^* + \dim S^\omega = \dim S + \dim S^\omega.
\end{equation*}
For proving (b), let $v \in S$. Then for any $u \in S^\omega$ we have that $\omega(v,u) = - \omega(u,v) = 0$ and thus $S \subseteq \del[1]{S^\omega}^\omega$. Hence $S$ is a linear subspace of $\del[1]{S^\omega}^\omega$. Furthermore part (a) yields
\begin{equation*}
\dim S = \dim V - \dim S^\omega = \dim \del[1]{S^\omega}^\omega
\end{equation*}
\noindent Thus exercise B.4. (b) \cite[620]{lee:smooth_manifolds:2013} implies that $\del[1]{S^\omega}^\omega = S$.\\
For (c), suppose that $S \subseteq T$ and let $v \in T^\omega$. Then for any $u \in S$ we have that $\omega(v,u) = 0$ and thus $T^\omega \subseteq S^\omega$. Conversly, suppose that $T^\omega \subseteq S^\omega$. By part (b) we can also show that $\del[1]{S^\omega}^\omega \subseteq \del[1]{T^\omega}^\omega$. But this holds as one can easily see. Thus $S \subseteq T$ and the statement follows.\\
For (d), we show the two equivalences separately. We have that
\begin{equation*}
\ker \widehat{\omega\vert_{S}} = \cbr[0]{v \in S : \forall w \in S\del[0]{\omega(v,w) = 0}} = S \cap S^\omega.
\end{equation*}
So $\omega\vert_{S}$ is nondegenerate if and only if $S \cap S^\omega = \cbr[0]{0}$. For the second equivalence, assume that $S \cap S^\omega = \cbr[0]{0}$. Then by \cite[100]{fischer:lineare_algebra:2014} and part (a) we have that
\begin{equation*}
\dim(S + S^\omega) = \dim S + \dim S^\omega - \dim(S \cap S^\omega) = \dim S + \dim S^\omega = \dim V.
\end{equation*}
Thus exercise B.4. (b) \cite[620]{lee:smooth_manifolds:2013} implies that $S + S^\omega = V$. Since $S \cap S^\omega = \cbr[0]{0}$ holds, we have $V = S \oplus S^\omega$ by \cite[101]{fischer:lineare_algebra:2014}. The other implication follows simply by definition of the direct sum.\\
(e) directly follows from (a) and \cite[620]{lee:smooth_manifolds:2013} since
\begin{equation*}
2\dim S \leq \dim S + \dim S^\omega = \dim V.
\end{equation*}
For (f) let $S$ have codimension $1$. Hence by part (a) we get that $\dim S^\omega = 1$. Thus any element in $S^\omega$ can be written as $\lambda v$, where $\lambda \in \Rbb$ and $v \in S^\omega \setminus \cbr[0]{0}$. Hence $\omega(\lambda v, \mu v) =  \lambda\mu \omega(v,v) = 0$ and thus $S^\omega \subseteq \del[1]{S^\omega}^\omega$ which is by part (b) equivalent to $S^\omega \subseteq S$. For proving (g), we first observe that the second equivalence is trivial. Now assume that $S$ is lagrangian. From part (a) immediately follows that $\dim S = \dim S^\omega$. Since $S \subseteq S^\omega$ we get that $S = S^\omega$. Conversly, assume that $S = S^\omega$. Using again part (a) we get that $2\dim S = \dim V$.
\end{solution}