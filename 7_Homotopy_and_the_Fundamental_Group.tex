\chapter{Homotopy and the Fundamental Group}
\section{The Fundamental Group}
\begin{exercise}
Let $X$ be a topological space. For any points $p,q \in X$, path homotopy is an equivalence relation on the set of all paths in $X$ from $p$ to $q$.
\end{exercise}

\begin{solution}
Let $f$ be a path in $X$ from $p$ to $q$. Define $H: I \times I \to X$ by $H(s,t) := f(s)$. Clearly $H$ is continuous since $f$ is. Indeed, take $U \subseteq X$ open, then $H^{-1}(U) = f^{-1}(U) \times I$ which is open in the box topology. Now clearly $H(s,0) = H(s,1) = f(s)$ for all $s \in I$ and $H(0,t) = f(0) = p$, $H(1,t) = f(1) = q$ for all $t \in I$. Hence $f \sim f$. If $f \sim g$ and $F$ is a path homotopy from $f$ to $g$, then $H: I \times I \to X$ defined by $H(s,t) := F(1-s,1-t)$ is a path homotopy from $g$ to $f$. Thus $g \sim f$. If $f \sim g$ and $g \sim h$ where $F$ and $G$ denote the path homotopies, respectively, then $H: I \times I \to X$ defined by
\begin{align*}
H(s,t) := \begin{cases}
F(s,2t) & 0 \leq t \leq \frac{1}{2},\\
G(s,2t - 1) & \frac{1}{2} \leq t \leq 1
\end{cases}
\end{align*}
\noindent is a path homotopy from $f$ to $h$, hence $f \sim g$.
\end{solution}

\begin{exercise}
Let $X$ be a path-connected topological space.
\begin{enumerate}[label = \textup{(}\alph*\textup{)}]
\item Let $f,g:I \to X$ be two paths from $p$ to $q$. Show that $f \sim g$ if and only if $f\overline{g} \sim c_p$.
\item Show that $X$ is simply connected if and only if any two paths in $X$ with the same initial and terminal points are path-homotopic.
\item Let $A \subseteq \mathbb{R}^n$ be convex. Then $A$ is simply connected.
\end{enumerate}
\end{exercise}

\begin{solution}
For (a), assume $f \sim g$. Hence $\eqclass{f} = \eqclass{g}$ and by the properties of path class products \cite[189]{lee:topological_manifolds:2011} we get
\begin{equation*}
\eqclass{f\overline{g}} = \eqclass{f}\eqclass{\overline{g}} = \eqclass{g} \eqclass{\overline{g}} = \eqclass{c_p}
\end{equation*} 
\noindent and thus $f\overline{g} \sim c_p$. Conversly, $f\overline{g} \sim c_p$ implies $\eqclass{f\overline{g}} = \eqclass{c_p}$ and thus
\begin{equation*}
\eqclass{g} = \eqclass{c_p}\eqclass{g} = (\eqclass{f\overline{g}}) \eqclass{g} = (\eqclass{f}\eqclass{\overline{g}})\eqclass{g} =\eqclass{f} (\eqclass{\overline{g}}\eqclass{g}) = \eqclass{f}\eqclass{c_q} = \eqclass{f}.
\end{equation*}
For (b), assume that $X$ is simply connected and let $f$ and $g$ be paths in $X$ from $p$ to $q$. Then $f\overline{g}$ is a loop based at $p$. Since $\pi_1(X,p) = \cbr[0]{\eqclass{c_p}}$, we get that $f\overline{g} \sim c_p$ and thus by part (a) that $f \sim g$. Conversly, let $f$ be a loop based at $p$. Hence $f \sim c_p$ and so $\pi_1(X,p)$ is trivial.
For (c), let $f$ and $g$ be paths in $A$ from $p$ to $q$. Then by example 7.4 \cite[185--186]{lee:topological_manifolds:2011} we get that $f \sim g$. Hence by part (b) follows that $A$ is simply connected.
\end{solution}

\begin{corollary}
$\mathbb{R}^n$ is simply connected.
\end{corollary}

\section{Categories and Functors}
See \cite[57--58]{lane:cat:1971}.
\begin{exercise}
Let $G$ be a group and $N \unlhd G$. Define $F: \cat{Grp} \to \cat{Set}$ by 
\begin{equation}
F(H) := \cbr[0]{f \in \Hom(G,H) : N \subseteq \ker f}.
\end{equation}
\begin{enumerate}
\item Show that $F$ is a functor.
\item Show that $\langle \sfrac{G}{N},\pi \rangle$ is a universal element of the functor $F$.
\end{enumerate}
\end{exercise}

\begin{solution}
For (i), we have to define first the action of $F$ on arrows of $\cat{Grp}$. Consider $A \xrightarrow{\varphi}{} B$. Define $F(\varphi) : F(A) \to F(B)$ by 
\begin{equation*}
F(\varphi)(f) := \varphi \circ f.
\end{equation*}
Let $f \in F(A)$. Then $F(\id_A)(f) = \id_A \circ f = f$ and thus $F(\id_A) = \id_{F(A)}$. Furthermore, for $B \xrightarrow{\psi}{} C$ we have that
\begin{align*}
F(\psi \circ \varphi)(f) &= (\psi \circ \varphi) \circ f\\
&= \psi \circ (\varphi \circ f)\\
&= \psi \circ F(\varphi)(f)\\
&= F(\psi)\del[1]{F(\varphi)(f)}\\
&= \del[1]{F(\psi) \circ F(\varphi)}(f)
\end{align*}
\noindent and so $F(\psi \circ \varphi) = F(\psi) \circ F(\varphi)$. Hence $F$ is a functor. For (ii), by proposition 4.7 \cite[20]{grillet:abstract_algebra:2007} we get that $\pi \in F(\sfrac{G}{N})$. Furthermore, consider $\langle A, \varphi \rangle$ for any $A$ object and $\varphi$ morphism in $\cat{Grp}$ such that $\varphi \in F(A)$. By the factorization theorem \cite[23]{grillet:abstract_algebra:2007} there exists a unique homomorphism $\psi: G \to A$ such that $\varphi = \psi \circ \pi$. Thus 
\begin{equation*}
F(\psi)(\pi) = \psi \circ \pi = \varphi
\end{equation*}
\noindent and thus $\langle \sfrac{G}{N},\pi \rangle$ is a universal element of the functor $F$.
\end{solution}

\begin{exercise}
Let $\cat{C}$ be a category and $(X_\alpha)_{\alpha \in A}$ be a familiy of objects of $\cat{C}$. If $(S,\iota_\alpha)$ and $(S',\iota'_\alpha)$ are two coproducts of $(X_\alpha)_{\alpha \in A}$, then there exists a unique isomorphism $f : S \to S'$ such that $f \circ \iota_\alpha = \iota'_\alpha$ for all $\alpha \in A$.
\end{exercise}

\begin{solution}
By the defining property of a coproduct there exist unique morphisms $f: S \to S'$ and $g: S' \to S$ as indicated in the commutative diagram below.
\begin{figure}[h!tb]
\begin{displaymath}
    	\xymatrix{ & S \ar[dr]^f\\
    		S' \ar[ur]^g & X_\alpha \ar[l]^{\iota_\alpha'}\ar[u]^{\iota_\alpha}\ar[r]_{\iota_\alpha'}\ar[d]_{\iota_\alpha} & S'\ar[dl]^g\\
    		& S}
\end{displaymath}
\end{figure}
\noindent Furthermore, above diagram yields
\begin{equation*}
(g \circ f) \circ \iota_\alpha = \iota_\alpha \qquad \text{and} \qquad (f \circ g) \circ \iota_\alpha' = \iota_\alpha'.
\end{equation*}
\noindent for all $\alpha \in A$ and thus the commutative diagrams
\begin{figure}[h!tb]
    \centering
    \begin{subfigure}[b]{0.3\textwidth}
        \begin{displaymath}
    			\xymatrix{ S \ar[dr]^{g \circ f}\\
    				X_\alpha \ar[u]^{\iota_\alpha}\ar[r]_{\iota_\alpha} & S }
		\end{displaymath}
    \end{subfigure}
    ~
    \begin{subfigure}[b]{0.3\textwidth}
        \begin{displaymath}
    			\xymatrix{ S' \ar[dr]^{f \circ g}\\
    				X_\alpha \ar[u]^{\iota_\alpha'}\ar[r]_{\iota_\alpha'} & S' }
		\end{displaymath}
    \end{subfigure}
\end{figure}
Also
\begin{figure}[h!tb]
    \centering
    \begin{subfigure}[b]{0.3\textwidth}
        \begin{displaymath}
    			\xymatrix{ S \ar[dr]^{\id_S}\\
    				X_\alpha \ar[u]^{\iota_\alpha}\ar[r]_{\iota_\alpha} & S }
		\end{displaymath}
    \end{subfigure}
    ~
    \begin{subfigure}[b]{0.3\textwidth}
        \begin{displaymath}
    			\xymatrix{ S' \ar[dr]^{\id_{S'}}\\
    				X_\alpha \ar[u]^{\iota_\alpha'}\ar[r]_{\iota_\alpha'} & S' }
		\end{displaymath}
    \end{subfigure}
\end{figure}
\noindent are commutative and so by the uniqueness property of the coproduct we get that 
\begin{equation*}
g \circ f = \id_S \qquad \text{and} \qquad f \circ g = \id_{S'}.
\end{equation*}
\end{solution}
