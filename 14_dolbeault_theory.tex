\chapter{Dolbeault Theory}
\section{Tensor Characterization Lemma}
\begin{definition}
Let $k,l \in \mathbb{Z}$, $k,l \geq 0$ and $M$ a smooth manifold. Then the \bld{bundle of mixed tensors of type $(k,l)$} is defined by
\begin{equation}
T^{(k,l)}TM := \coprod_{p \in M} T^{(k,l)}(T_pM).
\end{equation}
\end{definition}

\begin{proposition}
The bundle of mixed tensors of type $(k,l)$ has a unique natural structure as a smooth vector bundle of rank $n^{k + l}$ over $M$.
\label{prop:smooth_bundle}
\end{proposition}

\begin{proof}
For each $p \in M$ let $E_p := T^{(k,l)}(T_pM)$. By \cite[57]{lee:smooth_manifolds:2013} and \cite[313]{lee:smooth_manifolds:2013} $\dim E_p = n^{k + l}$. Furthermore, let $E := T^{(k,l)}TM$ and $\pi : E \to M$ be defined by $\pi(p,A) := p$. Let $\cbr[0]{(U_\alpha,\varphi_\alpha) : \alpha \in A}$ be an atlas for $M$. For each $\alpha \in A$ define
\begin{equation*}
\Phi_\alpha:\begin{cases}
\pi^{-1}(U_\alpha) \to U_\alpha \times \mathbb{R}^{n^{k + l}}\\
(p,A) \mapsto \del[1]{p,(A^{i_1\dots i_k}_{j_1\dots j_l})}
\end{cases}
\end{equation*}
Clearly, the inverse is given by
\begin{equation*}
\Phi^{-1}_\alpha:\begin{cases}
U_\alpha \times\mathbb{R}^{n^{k + l}} \to \pi^{-1}(U_\alpha)\\
\del[1]{p,(A^{i_1\dots i_k}_{j_1\dots j_l})} \mapsto (p,A)
\end{cases}.
\end{equation*}
Hence each $\Phi_\alpha$ is bijective. Now we have to check, that $\Phi_\alpha\vert_{E_p}$ is an isomorphism. So let $\lambda \in \mathbb{R}$ and $B \in E_p$. Then
\begin{align*}
\Phi_\alpha\vert_{E_p}(p,\lambda A + B) &= \del[1]{p,(\lambda A + B)^{i_1\dots i_k}_{j_1\dots j_l})}\\
&= \del[1]{p,\lambda (A^{i_1\dots i_k}_{j_1\dots j_l}) + (B^{i_1\dots i_k}_{j_1\dots j_l})}\\
&= \lambda\Phi_\alpha\vert_{E_p}(p,A) + \Phi_\alpha\vert_{E_p}(p,B).
\end{align*}
Now let $\alpha, \beta \in A$ such that $U_\alpha \cap U_\beta \neq \varnothing$. We consider the mapping
\begin{equation*}
\Phi_\alpha \circ \Phi_\beta^{-1} : (U_\alpha \cap U_\beta) \times \mathbb{R}^{n^{k + l}} \to (U_\alpha \cap U_\beta) \times \mathbb{R}^{n^{k + l}}.
\end{equation*}
Define $\tau_{\alpha\beta}: U_\alpha \cap U_\beta \to \GL(n^{k + l},\mathbb{R})$ by
\begin{equation*}
\tau_{\alpha\beta} := (\delta^i_j).
\end{equation*}
Then we have that
\begin{equation*}
(\Phi_\alpha \circ \Phi_\beta^{-1})\del[1]{p,(A^{i_1\dots i_k}_{j_1\dots j_l})} = \del[1]{p,(A^{i_1\dots i_k}_{j_1\dots j_l})} = \del[1]{p,\tau_{\alpha\beta}(p)(A^{i_1\dots i_k}_{j_1\dots j_l})}.
\end{equation*}
Since $\tau_{\alpha\beta}$ is constant, it is smooth (see \cite[36]{lee:smooth_manifolds:2013}). Hence we can apply the vector bundle chart lemma \cite[253]{lee:smooth_manifolds:2013} and the result follows.
\end{proof}

What follows is a reformulation of the smoothness criteria for tensor fields (\cite[317--318]{lee:smooth_manifolds:2013}) for tensor fields of type $(1,k)$.

\begin{proposition}[Smoothness Criteria for Tensor Fields]
Let $M$ be smooth manifold and let $A : M \to T^{(1,k)}TM$ be a rough section. Then the following are equivalent:
\begin{enumerate}[label = \textup{(}\alph*\textup{)}]
\item $A \in \Gamma(T^{(1,k)}TM)$.
\item In every smooth coordinate chart, the component functions of $A$ are smooth.
\item For all $X_1,\dots,X_k \in \Xfrak(M)$, the rough section $A(X_1,\dots,X_k) : M \to TM$ defined by
\begin{equation}
A(X_1,\dots,X_k)(p) := A_p(X_1\vert_p,\dots,X_k\vert_p)
\end{equation}
\noindent is a smooth vector field.
\item If $X_1,\dots,X_k$ are smooth vector fields on some open subset $U \subseteq M$, then also $A(X_1,\dots,X_k)$ is a smooth vector field on $U$.
\end{enumerate}
\label{prop:smoothness_tensor}
\end{proposition}

\begin{proof}
We prove (a)$\Leftrightarrow$(b) and (b)$\Rightarrow$(c)$\Rightarrow$(d)$\Rightarrow$(b).\\
To prove (a)$\Leftrightarrow$(b), let $(U,(x^i))$ be a smooth chart. Actually, we can prove this for general tensor fields of type $(k,l)$. Proposition \ref{prop:smooth_bundle} together with the proof of the vector bundle chart lemma \cite[253--254]{lee:smooth_manifolds:2013} implies, that the corresponding chart on $T^{(k,l)}TM$ is given by $\del[1]{\pi^{-1}(U),\widetilde{\varphi}}$, where $\widetilde{\varphi}: \pi^{-1}(U) \to \varphi(U) \times \mathbb{R}^{n^{k+l}}$ is defined by
\begin{equation*}
\widetilde{\varphi} := (\varphi \times \id_{\mathbb{R}^{n^{k+l}}}) \circ \Phi
\end{equation*}
\noindent where $\Phi : \pi^{-1}(U) \to U \times \mathbb{R}^{n^{k + l}}$ is given as in the proof of proposition \ref{prop:smooth_bundle}. Now we consider the coordinate representation $\widehat{A}$ in the given charts (see \cite[35]{lee:smooth_manifolds:2013}). Since $A$ is a rough section, we have that 
\begin{equation*}
A^{-1}\del[1]{\pi^{-1}(U)} = (\pi \circ A)^{-1}(U) = \id_M^{-1}(U) = U.
\end{equation*}
Hence $\varphi\del[1]{U \cap A^{-1}(\pi^{-1}(U))} = \varphi(U)$, which is open, and $\widehat{A}: \varphi(U) \to \widetilde{\varphi}\del[1]{\pi^{-1}(U)}$ is given by
\begin{align*}
\widehat{A}(x) &= (\widetilde{\varphi} \circ A \circ \varphi^{-1})(x)\\
&= (\varphi \times \id_{\mathbb{R}^{n^{k+l}}})\del[1]{\Phi(\varphi^{-1}(x),A_{\varphi^{-1}(x)})}\\
&= (\varphi \times \id_{\mathbb{R}^{n^{k+l}}})\del[1]{\varphi^{-1}(x),\del[1]{A^{i_1\dots i_k}_{j_1\dots j_l}(\varphi^{-1}(x))}}\\
&= \del[1]{x,\del[1]{\widehat{A}^{i_1\dots i_k}_{j_1\dots j_l}(x)}}.
\end{align*}
By \cite[35]{lee:smooth_manifolds:2013} $A$ is smooth if and only if in any chart $\widehat{A}$ is smooth. This is furthermore equivalent to that each $\widehat{A}^{i_1\dots i_k}_{j_1\dots j_l}$ is smooth and thus equivalent to that $A^{i_1\dots i_k}_{j_1\dots j_l}$ is smooth (see \cite[33]{lee:smooth_manifolds:2013}).\\
To prove (b)$\Rightarrow$(c), let $(U,(x^i))$ be a smooth chart. Then write $X_1,\dots,X_k \in \Xfrak(M)$ as 
\begin{equation*}
X_\nu = X^{\mu_\nu}_\nu \frac{\partial}{\partial x^{\mu_\nu}}.
\end{equation*}
\noindent for $\nu = 1,\dots,k$. For $p \in U$ lemma \ref{lem:corr} implies
\begin{align*}
A(X_1,\dots,X_n)(p) &= A_p(X_1\vert_p,\dots,X_k\vert_p)\\
&= A_p\del[3]{X^{\mu_1}_1(p) \frac{\partial}{\partial x^{\mu_1}}\bigg\vert_p,\dots,X^{\mu_k}_1(p) \frac{\partial}{\partial x^{\mu_k}}\bigg\vert_p}\\
&= X^{\mu_1}_1(p) \cdots X^{\mu_k}_k(p)A_p\del[3]{\frac{\partial}{\partial x^{\mu_1}}\bigg\vert_p,\dots,\frac{\partial}{\partial x^{\mu_k}}\bigg\vert_p}\\
&= X^{\mu_1}_1(p) \cdots X^{\mu_k}_k(p) A^i_{\mu_1 \dots \mu_k}(p)\frac{\partial}{\partial x^i}\bigg\vert_p.
\end{align*}
By the smoothness criterion for vector fields \cite[175]{lee:smooth_manifolds:2013} we have that each component function $X^{\mu_n}_\nu$ is smooth. Thus if $A$ is smooth, we have by  that each $A^i_{j_1\dots j_k}$ is smooth and since $\Cscr^\infty(M)$ is an $\mathbb{R}$-algebra (see \cite[33]{lee:smooth_manifolds:2013}), we have that 
\begin{equation*}
X^{\mu_1}_1 \cdots X^{\mu_k}_k A^i_{\mu_1 \dots \mu_k}
\end{equation*} 
\noindent is smooth for $i = 1,\dots,n$. Thus again by the smoothness criterion together with the localness of smoothness \cite[35]{lee:smooth_manifolds:2013} we get that $A(X_1,\dots,X_k) \in \Xfrak(M)$.\\
To prove (c)$\Rightarrow$(d), we use that smoothness is a local property (see \cite[35]{lee:smooth_manifolds:2013}). Let $p \in U$.  Then by \cite[14]{cattaneo:manifolds:2017} we find a smooth bump function $\psi$ supported in $U$ and identically equal to $1$ on some neighbourhood $V$ of $p$. Set 
\begin{align*}
\widetilde{X}_\nu\vert_p := \begin{cases}
\psi(p) X_\nu\vert_p & p \in \supp \psi\\
0 & p \in M \setminus \supp \psi
\end{cases}.
\end{align*}
Then the gluing lemma for smooth maps \cite[35]{lee:smooth_manifolds:2013} implies $\widetilde{X}_1,\dots,\widetilde{X}_k \in \Xfrak(M)$. Hence by (c) we get that $A(\widetilde{X}_1,\dots,\widetilde{X}_k) \in \Xfrak(M)$ and so the restriction $A(\widetilde{X}_1,\dots,\widetilde{X}_k)\vert_V$ is smooth. But $A(\widetilde{X}_1,\dots,\widetilde{X}_k)\vert_V = A(X_1,\dots,X_k)$ and so we are done.\\
Lasty to prove (d)$\Rightarrow$(b), each vector field locally defined by 
\begin{equation*}
X_{j_\nu} = \delta^{\mu_\nu}_{j_\nu} \frac{\partial}{\partial x^{\mu_\nu}}.
\end{equation*}
\noindent is smooth. Thus by
\begin{equation*}
A(X_1,\dots,X_n)(p) = \delta^{\mu_1}_{j_1} \cdots \delta^{\mu_k}_{j_k} A^i_{\mu_1 \dots \mu_k}(p)\frac{\partial}{\partial x^i}\bigg\vert_p = A^i_{j_1\dots j_k}(p)\frac{\partial}{\partial x^i}\bigg\vert_p
\end{equation*}
\noindent we get that $A^i_{j_1 \dots j_k}$ is smooth and hence by (b) also $A$.
\end{proof}

\begin{theorem}[Tensor Characterization Lemma]
A mapping
\begin{equation*}
\underbrace{\mathfrak{X}(M) \times \dots \times \mathfrak{X}(M)}_{k} \to \Cscr^\infty(M) \qquad \text{or} \qquad \underbrace{\mathfrak{X}(M) \times \dots \times \mathfrak{X}(M)}_{k} \to \Xfrak(M)
\end{equation*}
\noindent is induced by an element of $\Gamma(T^{(0,k)}TM)$ or $\Gamma(T^{(1,k)}TM)$, respectively, if and only if they are multilinear over $\Cscr^\infty(M)$.
\label{thm:tensor_characterization_lemma}
\end{theorem}

\begin{proof}
We are proving only the second statement. Any element in $\Gamma(T^{(1,k)}TM)$ induces a mapping $\mathfrak{X}(M) \times \dots \times \mathfrak{X}(M) \to \Xfrak(M)$ by part (c) of the smoothness criteria for tensor fields \ref{prop:smoothness_tensor}. Thus we have to show that $\Ascr$ is multilinear over $\Cscr^\infty(M)$. Let $f \in \Cscr^\infty(M)$ and $X_\nu,\widetilde{X}_\nu \in \mathfrak{X}(M)$, $\nu = 1,\dots,k$. Then for any $p \in M$ we have that
\begin{align*}
\Ascr(X_1,\dots,fX_\nu + \widetilde{X}_\nu,\dots,X_k)_p =& A_p(X_1\vert_p,\dots,(fX_\nu + \widetilde{X}_\nu)_p,\dots,X_k\vert_p)\\
=& A_p(X_1\vert_p,\dots,f(p)X_\nu\vert_p + \widetilde{X}_\nu\vert_p,\dots,X_k\vert_p)\\
=& f(p)A_p(X_1\vert_p,\dots,X_\nu\vert_p ,\dots,X_k\vert_p)\\
& + A_p(X_1\vert_p,\dots,\widetilde{X}_\nu\vert_p,\dots,X_k\vert_p)\\
=& f(p) \Ascr(X_1,\dots,X_\nu,\dots,X_k)_p\\
& + \Ascr(X_1,\dots,\widetilde{X}_\nu,\dots,X_k)_p\\
=& \del[1]{f \Ascr(X_1,\dots,X_\nu,\dots,X_k)}_p \\
& + \Ascr(X_1,\dots,\widetilde{X}_\nu,\dots,X_k)_p.
\end{align*}
Conversly, suppose that $\Ascr: \mathfrak{X}(M) \times \dots \times \mathfrak{X}(M) \to \Xfrak(M)$ is multilinear over $\Cscr^\infty(M)$. Let $p \in M$. First we show that $\Ascr$ acts locally, i.e. if $X_\nu = \widetilde{X}_\nu$ in some neighbourhood $U$ of $p$ implies that also 
\begin{equation*}
\Ascr(X_1,\dots,X_\nu,\dots,X_k) = \Ascr(X_1,\dots,\widetilde{X}_\nu,\dots,X_k)
\end{equation*}
\noindent on $U$. By the multilinearity of $\Ascr$ it is enough to show that if $X_\nu$ vanishes on $U$ then so does $\Ascr$. There exists a smooth bump function $\psi$ for $\cbr[0]{p}$ supported in $U$ (see \cite[44]{lee:smooth_manifolds:2013}). Hence $\psi X_\nu = 0$ on $M$ and $\psi(p) = 1$. Thus
\begin{align*}
0 = \Ascr(X_1,\dots,\psi X_\nu,\dots,X_k)_p = \psi(p)\Ascr(X_1,\dots,X_\nu,\dots,X_k)_p.
\end{align*}
\noindent and since $\psi(p) = 1$ we have that
\begin{equation*}
\Ascr(X_1,\dots,X_\nu,\dots,X_k)_p = 0
\end{equation*}
\noindent for any $p \in U$.\\
Next we show that $\Ascr$ actually acts pointwise, i.e. if $X_\nu\vert_p$ vanishes so does $\Ascr$. Let $(U,(x^i))$ be a chart containing $p$ and $X_\nu = X_\nu^i \frac{\partial}{\partial x^i}$ on $U$. The same construction as used showing the implication (c)$\Rightarrow$(d) in the proof of proposition \ref{prop:smoothness_tensor} yields the existence of $f^1,\dots,f^n \in \Cscr^\infty(M)$ and $\widetilde{X}_1,\dots,\widetilde{X}_n \in \Xfrak(M)$ such that $f^i = X_\nu^i$ and $\widetilde{X}_i = \frac{\partial}{\partial x^i}$ on a neighbourhood $V \subseteq U$ of $p$. Thus by the previous localization, we get that 
\begin{equation*}
\Ascr(X_1,\dots,X_\nu,\dots,X_k) = \Ascr(X_1,\dots,f^i\widetilde{X}_i,\dots,X_k) = f^i\Ascr(X_1,\dots,\widetilde{X}_i,\dots,X_k) 
\end{equation*} 
\noindent in $U$. Since $0 = X_\nu^i(p) = f^i(p)$, $\Ascr$ vanishes at $p$. Hence $\Ascr$ depends only on the value of $X_\nu$ at $p$. Thus define a rough section $A : M \to T^{(1,k)}TM$ by 
\begin{equation*}
A_p(v_1,\dots,v_k) := \Ascr(V_1,\dots,V_k)(p)
\end{equation*}
\noindent where $V_1,\dots,V_k \in \Xfrak(M)$ are any extensions of $v_1,\dots,v_k \in T_pM$ (see \cite[177]{lee:smooth_manifolds:2013}). By the above, the choice of the extensions does not matter and the resulting rough section is smooth by proposition \ref{prop:smoothness_tensor} part (c), hence $A \in \Gamma(T^{(1,k)}TM)$.
\end{proof}

\section{Integrability}

\begin{definition}
Let $(M,J)$ be an almost complex manifold. For $X,Y \in \Xfrak(M)$ define the \bld{Nijenhuis tensor $N$} as
\begin{equation}
N(X,Y) := \sbr[0]{JX,JY} - J\sbr[0]{X,JY} - J\sbr[0]{JX,Y} - \sbr[0]{X,Y}
\end{equation}
\noindent where $\sbr[0]{X,Y}$ denotes the usual Lie-bracket of vector fields.
\end{definition}

\begin{exercise}
~
\begin{enumerate}[label = \textup{(}\alph*\textup{)}]
\item If $\sbr[0]{\cdot,Y} \circ J = J \circ \sbr[0]{\cdot,Y}$ for all $Y \in \Xfrak(M)$, then $N(X,Y) = 0$ for all $X \in \Xfrak(M)$.
\item $N$ is actually a tensor.
\end{enumerate}
\end{exercise}

\begin{solution}
Part (a) simply follows from
\begin{equation*}
N(X,Y) = J\sbr[0]{X,JY} - J\sbr[0]{X,JY} + \sbr[0]{X,Y} - \sbr[0]{X,Y} = 0.
\end{equation*}
To prove (b), we observe that by the tensor characterization lemma \ref{thm:tensor_characterization_lemma} it is enough to show that $N$ is multilinear over $\Cscr^\infty(M)$. Let $f \in \Cscr^\infty(M)$ and $X,Y,Z \in \mathfrak{X}(M)$. Then
\begin{align*}
N(fX + Y,Z) =& \sbr[0]{J(fX + Y),JZ} - J\sbr[0]{fX + Y,JZ} - J\sbr[0]{J(fX + Y),Z}\\
 &- \sbr[0]{fX + Y,Z}\\
=& \sbr[0]{fJX + JY,JZ} - J\sbr[0]{fX + Y,JZ} - J\sbr[0]{fJX + JY,Z}\\
& - \sbr[0]{fX + Y,Z}\\
=& \sbr[0]{fJX,JZ} + \sbr[0]{JY,JZ} - J\sbr[0]{fX,JZ} - J\sbr[0]{Y,JZ} - J\sbr[0]{fJX,Z}\\
& -J\sbr[0]{JY,Z} - \sbr[0]{fX,Z} - \sbr[0]{Y,Z}\\
=& f\sbr[0]{JX,JZ} - (JZf)JX + \sbr[0]{JY,JZ} - fJ\sbr[0]{X,JZ} + (JZf)JX\\
& - \sbr[0]{Y,JZ} - fJ\sbr[0]{JX,Z} + (Zf)JJX - J\sbr[0]{JY,Z} - f\sbr[0]{X,Z}\\
& + (Zf)X - \sbr[0]{Y,Z}\\
=& fN(X,Z) + N(Y,Z).
\end{align*}
\noindent by \cite[187--188]{lee:smooth_manifolds:2013}. Linearity in the second argument is shown similarly. Hence $N : \mathfrak{X}(M) \times \mathfrak{X}(M) \to \mathfrak{X}(M)$ is bilinear over $\Cscr^\infty(M)$.
\end{solution}