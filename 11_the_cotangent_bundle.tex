\chapter{The Cotangent Bundle}
\section{Line Integrals}
\subsection{The Winding Number*}
\begin{definition}[Winding Number]
Let $z_0 \in \mathbb{C}$ and $\gamma: \intcc{a,b} \to \mathbb{C} \setminus \cbr[0]{z_0}$ be a piecewise continuously differentiable closed path. Then
\begin{equation}
\W(\gamma,z_0) := \frac{1}{2\pi i} \int_\gamma \frac{\d z}{z - z_0}
\end{equation}
\noindent is called the \bld{winding number} of $\gamma$ around $z_0$.
\label{def:winding_number}
\end{definition}

\begin{proposition}
Let $z_0 := x_0 + iy_0 \in \mathbb{C}$ and $\gamma: \intcc{a,b} \to \mathbb{C} \setminus \cbr[0]{z_0}$ be a piecewise continuously differentiable closed path. Then 
\begin{equation}
\W(\gamma,z_0) = \int_\gamma \omega
\end{equation}
\noindent where $\omega \in \Omega^1\del[1]{\mathbb{R}^2 \setminus \cbr[0]{z_0}}$ is given by
\begin{equation}
\omega := \frac{1}{2\pi}\frac{(x - x_0)\d y - (y - y_0) \d x}{(x - x_0)^2 + (y - y_0)^2}.
\end{equation}
\label{prop:winding_number_equivalence}
\end{proposition}

\begin{proof}
This immediately follows from 
\begin{align*}
\int_\gamma \frac{\d z}{z - z_0} &= \int_\gamma \frac{\d x + i \d y}{(x + iy) - (x_0 + iy_0)}\\
&= \int_\gamma \frac{\d x + i \d y}{(x - x_0) + i(y - y_0)}\frac{(x - x_0) - i(y - y_0)}{(x - x_0) - i(y - y_0)}\\
&= \int_\gamma \frac{(x - x_0)\d x + \del[1]{(x - x_0)\d y - (y - y_0)\d x} + (y - y_0)\d y}{(x - x_0)^2 + (y - y_0)^2}\\
&= \int_\gamma \frac{(x - x_0)\d x + (y - y_0)\d y}{(x - x_0)^2 + (y - y_0)^2} + i\int_\gamma \frac{(x - x_0)\d y - (y - y_0)\d x}{(x - x_0)^2 + (y - y_0)^2}\\
&= i\int_\gamma \frac{(x - x_0)\d y - (y - y_0)\d x}{(x - x_0)^2 + (y - y_0)^2}
\end{align*}
\noindent by the fundamental theorem for line integrals \cite[291]{lee:smooth_manifolds:2013} since 
\begin{equation*}
\d \del[3]{\frac{1}{2} \log\del[1]{(x - x_0)^2 + (y - y_0)^2}} = \frac{(x - x_0)\d x + (y - y_0)\d y}{(x - x_0)^2 + (y - y_0)^2}.
\end{equation*}
\end{proof}

\begin{remark}
By proposition \ref{prop:winding_number_equivalence}, the definition of the winding number in complex analysis given by definition \ref{def:winding_number} coincides with the one usually given in algebraic topology (see for example \cite[19--20]{fulton:algebraic_topology:1995}).
\end{remark}