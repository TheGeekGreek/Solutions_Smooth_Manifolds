\chapter{Tensors}
\section{Multilinear Algebra}
We follow the terminology established in \cite[312]{lee:smooth_manifolds:2013}. 
\begin{definition}
Let $V$ be a finite-dimensional real vector space and $k,l \in \mathbb{Z}$ where $k,l \geq 0$. Then we define the \bld{space of mixed tensors of type $(k,l)$ on $V$} by
\begin{equation}
T^{(k,l)}(V) := \underbrace{V \otimes \dots \otimes V}_{k} \otimes \underbrace{V^* \otimes \dots \otimes V^*}_{l}
\end{equation}
\noindent if $(k,l) \neq (0,0)$ and
\begin{equation}
T^{(0,0)}(V) := \mathbb{R}
\end{equation}
\noindent otherwise.
\end{definition}

\begin{proposition}[Tensor Characterization Lemma]
Let $V$ be a finite-dimensional real vector space and $k,l \in \mathbb{Z}$ where $k \geq 1$, $l \geq 0$ and $(k,l) \neq (1,0)$. Then 
\begin{equation}
\boxed{T^{(k,l)}(V) \cong \Lrm\del[1]{(V^*)^{k - 1},V^l;V}.}
\end{equation}
\end{proposition}

\begin{lemma}

\end{lemma}

\begin{proof}
Define
\begin{equation*}
\Phi : V^k \times (V^*)^l \to \Lrm\del[1]{(V^*)^{k - 1},V^l;V}
\end{equation*}
\noindent by letting
\begin{equation*}
\Phi(v,\varphi)(\psi,w) := \varphi_1(w_1) \cdots \varphi_l(w_l)\psi_1(v_1) \cdots \psi_{k-1}(v_{k-1})v_k.
\end{equation*}
It is easyily checked that $\Phi(v,\varphi) \in \Lrm\del[1]{(V^*)^{k - 1},V^l;V}$ and that $\Phi$ is multilinear. By the characteristic property of the tensor product space \cite[309]{lee:smooth_manifolds:2013} there exists a unique linear mapping 
\begin{equation*}
\widetilde{\Phi} : V^{\otimes k} \otimes (V^*)^{\otimes l} \to \Lrm\del[1]{(V^*)^{k - 1},V^l;V}
\end{equation*}
\noindent such that 
\begin{equation*}
\Phi = \widetilde{\Phi} \circ \pi.
\end{equation*}
Now we claim that $\ker\widetilde{\Phi} = \cbr{0}$. Let $v \otimes \varphi \in \ker\widetilde{\Phi}$ and assume that $v,\varphi \neq 0$. Hence we find $w \in V^l$ such that $\varphi_i(w_i) \neq 0$ for all $i = 1,\dots,l$. Furthermore since $v_1,\dots,v_k \neq 0$, we find $\psi \in (V^*)^{k-1}$ such that $\psi_i(v_i) \neq 0$ for all $i = 1,\dots,k-1$. For example, if $(e_j)$ is a basis of $V$ then $v_i = r^j_i e_i$ where at least one $r_i^j \neq 0$, say $r_i^k$. Then let $\psi_i := e^*_k$ where $(e_j^*)$ denotes the corresponding basis of $V^*$. Then
\begin{equation*}
\widetilde{\Phi}(v,\varphi)(\psi,w) = \varphi_1(w_1) \cdots \varphi_l(w_l)\psi_1(v_1) \cdots \psi_{k-1}(v_{k-1})v_k \neq 0.
\end{equation*}
Contradiction. Thus the claim holds and we get that $\widetilde{\Phi}$ is injective. Since 
\begin{equation*}
\dim \del[1]{V^{\otimes k} \otimes (V^*)^{\otimes l}} = (\dim V)^{k + l} = \dim \del[1]{\Lrm\del[1]{(V^*)^{k - 1},V^l;V}}
\end{equation*}
\noindent by \cite[309]{lee:smooth_manifolds:2013}
\end{proof}


\section{Pullbacks of Tensor Fields}

\begin{exercise}[Properties of Tensor Pullbacks]
	Suppose $F: M \to N$ is a smooth mapping and $A$, $B$ are covariant tensor fields on $N$. Then
	\begin{enumerate}[label = (\alph*)]
		\item $F^*(A \otimes B) = F^*A \otimes F^* B$.
	\end{enumerate}
\end{exercise}

\begin{solution}
	Let $p \in M$. Then we have
	\begin{align*}
		\del[1]{F^*(A\otimes B)}_p (v_1,\dots,v_{k + l})&= (A \otimes B)_{F(p)}\del[1]{\d F_p(v_1), \dots, \d F_p(v_{k + l})}\\
		&= \del[1]{A_{F(p)} \otimes B_{F(p)}} \del[1]{\d F_p(v_1), \dots, \d F_p(v_{k + l})}\\
		&= A_{F(p)}\del[1]{\d F_p(v_1), \dots, \d F_p(v_{k})}B_{F(p)}\del[1]{\d F_p(v_{k+1}), \dots, \d F_p(v_{k + l})}\\
		&= (F^* A)_p(v_1,\dots,v_k) (F^* B)_p(v_{k+1},\dots,v_{k + l})\\
		&= \del[1]{(F^* A)_p \otimes (F^* B)_p}(v_1,\dots,v_{k + l})\\
		&= (F^* A \otimes F^* B)_p(v_1,\dots,v_{k + l})
	\end{align*}
	\noindent for all $v_1,\dots,v_{k + l} \in T_pM$.

\end{solution}
